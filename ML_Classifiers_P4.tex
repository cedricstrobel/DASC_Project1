1) Feature scaling for gradient based algorithms is important because of
    - Features might contribute differently depending on there scale which 
    causes false predictions
    - Standardization allows a more stable and faster optimization because 
    the algorithm wont jump in different step sizes over the values and 
    rather drives towords the optimum more smoothly

2) Batch Gradient Descent 
    - uses entire dataset to compute gradients which makes it expensive for large ones but 
    ideal for smaller ones
    - ideal for convex problems where global minima are guaranteed

    Stochastic Gradient Descent
    - uses only one random sample which is faster and more 
    suitable for large datasets
    - is robust because wont get stuck in local minima but 
    might miss the optimum because of two large step sizes

3) Why does scikit-learn Perceptron and Adaline outperform book code?
    - The learning rate is fix in the book code, while scikit-learn 
    uses time-based adaptive methods, which stabilizes late training
    - Scikit-learn supports early stopping which performes better 
    timewise and avoids overfitting
    - The bookcode has no regularization, so no L2/L1 penalties which 
    allows weights to grow too much which leads to overfitting
    - There is weight averaging in scikit-learn which reduzes noise 
    and allows a better generalization
    
4) While the LR decision boundary is linear and corresponds to the points 
    where the model predicts a probability of 0.5, the SVM can have non
    linear boundaries with the use of kernels and its purpose is to 
    maximize the margin between the two classes. 
    A differece is that outliners can affect LR boundaries strongly, while 
    they have less impact on SVM boundaries unless they lie within the margin.
    Finally the result differs too, LR will give probabilities, while 
    SVM only provides hard decision boundaries as classification

5) L1 and L2 add constraints on model complexity by penalizing large weights
        -L1 (Lasso): Encourages sparsity -> Feature selection
        -L2 (Ridge): Shrinks weights -> Prevents dominance of 
            specific Features
    - It prevents the model to overfit by preventing fitting on noise 
    and improves generalization

6) The C parameter controls the strength of regularization, where 
    a small value describes strong regularization and a high value weak 
    regularization. 
    In action it shows that a higher C value leads to a lower mean-test-score.
    For both models C=100 performed the best. For 200 quite similar but 
    drops for higher values.
